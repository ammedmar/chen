\immediate\write18{biber \jobname}

\documentclass{amsart}
\input{aux/style}
\input{aux/usualcmds}
\addbibresource{aux/usualpapers.bib}

%%%%%%%%%%%%%%%%%%%%%%
% !TEX root = ../chen.tex

\newcommand{\union}{\cup}
\renewcommand{\cup}{\smallsmile}
                   % add commands here
\addbibresource{aux/bibliography.bib}  % add references here
%\usepackage{package}                  % add packages here
\usepackage{parskip}

%%%%%%%%%%%%%%%%%%%%%%
\title[Short title]{Full title}

%\author[A.~Medina-Mardones]{Anibal~M.~Medina-Mardones}
%\address{A.M-M., Universit\'e Sorbonne Paris Nord \and Max Planck Institute for Mathematics}
%\email{\href{mailto:medina-mardones@math.univ-paris13.fr}{medina-mardones@math.univ-paris13.fr}}

%\date{\today}
%\subjclass[2020]{TBW}
%\keywords{TBW}

\begin{document}
\thispagestyle{empty}
Define for $i \in \N$ the linear map $\rC_i \colon A^{\ot 4} \to A$ by
\[
\rC_i(\alpha \ot \alpha' \ot \beta \ot \beta') \ =\
(\alpha \cup_0 \beta) \cup_i (\alpha' \cup_0 \beta') \ +
\sum_{i=j+k} (\alpha \cup_j \alpha') \cup_0 (12)^j(\beta \cup_k \beta').
\]
When $\alpha = \alpha'$ and $\beta = \beta'$ this expression is a lift of the Cartan relation to the cochain level.

Using the notation of \cite{medina2020cartan}, recall that a Cartan coboundary for cocycles $\alpha$ and $\beta$ is given by
\[
\zeta_i(\alpha,\beta) = H(\tilde x_i)(\alpha \ot \alpha \ot \beta \ot \beta),
\]
where $H \colon \cE(2) \to \cE(4)$ is an explicitly constructed homotopy between certain maps $(23)F$ and $G$ satisfying $(12)(34) \circ H = H \circ (12)$.
Explicitly,
\begin{equation}\label{e:homotopy}
	\bd \circ \, H + H \! \circ \bd = (23)F + G.
\end{equation}
The importance of this is that, for $i\in\N$, the image of $\big((23)F + G\big)(\tilde{x}_i)$ via the given (and implicit) chain map $\cE(4) \to \Hom(A^{\ot 4}, A)$ is $\rC_i$.
We will abuse notation and infer the presence of this latter map from the context.
After evaluating \cref{e:homotopy} on $\tilde x_i$ we have
\begin{equation}\label{e:homotopy evaluated}
	\delta \circ H(\tilde x_i) + H(\tilde x_i) \circ \delta + \big(1+(12)(34)\big) H(\tilde x_{i-1}) = \rC_i
\end{equation}
in $\Hom(A^{\ot 4}, A)$.
Notice that $\big(1+(12)(34)\big)(\alpha \ot \alpha \ot \beta \ot \beta) = 0$.

Let
\[
\gamma = \delta\alpha \ot \alpha \ot \beta \ot \beta + \alpha \ot \alpha \ot \delta\beta \ot \beta.
\]
Then
\begin{equation}
\begin{split}
	H(\tilde x_i) \circ \delta (\alpha \ot \alpha \ot \beta \ot \beta) & =
	H(\tilde x_i) \big(1 + (12)(34)\big)(\gamma) \\ & =
	H (\bd \tilde x_{i+1})(\gamma) \\ & =
	\big(\delta \circ H(\tilde x_{i+1}) + H(\tilde x_{i+1}) \circ \delta + \rC_{i+1}\big) (\gamma).
\end{split}
\end{equation}

We have that
\begin{align*}
	\delta\gamma &=
	\delta\alpha \ot \delta\alpha \ot \beta \ot \beta \ +\
	\alpha \ot \alpha \ot \delta\beta \ot \delta\beta \\ &+
	\big(1+(12)(34)\big)(\delta\alpha \ot \alpha \ot \beta \ot \delta\beta).
\end{align*}
Also, as before,
\begin{equation*}
	\big(1+(12)(34)\big)H(\tilde{x}_{i+1}) =
	\delta \circ H(\tilde x_{i+2}) + H(\tilde x_{i+2}) \circ \delta + \rC_{i+2}.
\end{equation*}
Putting everything together after skipping some similar steps we have
\begin{multline*}
	(\delta H_i + \rC_i)(\alpha \alpha \beta \beta) +
	(\delta H_{i+1} + \rC_{i+1})(\delta\alpha \alpha \beta \beta + \alpha \alpha \delta\beta \beta) +
	(\delta H_{i+2} + \rC_{i+2})(\delta\alpha \alpha \beta \delta\beta) \\ +\
	H_{i+1}(\delta\alpha \delta\alpha \beta \beta + \alpha \alpha \delta\beta \delta\beta) +
	H_{i+2}(\delta\alpha \delta\alpha \beta \delta\beta + \delta\alpha \alpha \delta\beta \delta\beta) = 0,
\end{multline*}
where $H_k$ is short for $H(\tilde{x}_k)$ and we are omitting tensor products and composition symbols.
If both $\alpha$ and $\beta$ are cocycles this expression recovers the main contribution of \cite{medina2020cartan}.

\sloppy
\printbibliography
\end{document}
